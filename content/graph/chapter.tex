\chapter{Graph}

% \newcommand\Inn{%
%   \mathrel{\ooalign{$\subset$\cr\hfil\scalebox{0.8}[1]{$=$}\hfil\cr}}%
% }

\section{Fundamentals}
	\kactlimport{BellmanFord.h}
	\kactlimport{FloydWarshall.h}
	\kactlimport{TopoSort.h}

\section{Network flow}
	% \kactlimport{PushRelabel.h}
	\kactlimport{MinCostMaxFlow.h}
	% \kactlimport{EdmondsKarp.h}
	\kactlimport{Dinic.h}
	\subsection{Flows with Demands(Lower Bound on flows)}
		% Source: https://cp-algorithms.com/graph/flow_with_demands.html
		% Test: NULL
		Finding an Arbitrary flow:
		We make the following changes in the network. We add a new source s' and 
		a new sink	t', a new edge from the source s' to every other vertex, a 
		new edge for every vertex to the sink t', and one edge from t to s. 
		Additionally we define the new capacity function c' as:
		$$c'((s', v)) = \sum_{u\in V} d((u, v)) \forall (s', v)$$
		$$c'((v, t')) = \sum_{w\in V} d((v, w)) \forall (v, t')$$
		$$c'((u, v)) = c((u, v)) - d((u, v)) \forall (u, v)$$ in the old network
		$$c'((t, s)) = INF$$

		Minimal Flow:
		The edge (t, s) flows the entire flow of the corresponding old network, and
		if we keep the c'((t, s)) to be INF, it will have the same flow of old network,
		but if we reduce the capacity to a very small value, the demands will not meet.
		So we can do a binary search on capacity of this edge and try to reduce it as much
		as possible so that demands meet.

	\kactlimport{MinCut.h}
	\kactlimport{GlobalMinCut.h}
	\kactlimport{GomoryHu.h}

\section{Matching}
	\kactlimport{hopcroftKarp.h}
	\kactlimport{DFSMatching.h}
	\kactlimport{MinimumVertexCover.h}
	\kactlimport{WeightedMatching.h}
	% \kactlimport{GeneralMatching.h}
	\kactlimport{GeneralMatchingBlossom.h}

\section{DFS algorithms}
	\kactlimport{SCC.h}
	\kactlimport{BiconnectedComponents.h}
	\kactlimport{2sat.h}
	\kactlimport{EulerWalk.h}

\section{Coloring}
	\kactlimport{EdgeColoring.h}

\section{Heuristics}
	\kactlimport{MaximalCliques.h}
	\kactlimport{MaximumClique.h}
	\kactlimport{MaximumIndependentSet.h}

\section{Trees}
	\kactlimport{BinaryLifting.h}
	\kactlimport{LCA.h}
	\kactlimport{CompressTree.h}
	\kactlimport{HLD.h}
	\kactlimport{LinkCutTree.h}
	\kactlimport{DirectedMST.h}

\section{Math}
	\subsection{Number of Spanning Trees}
		% I.e. matrix-tree theorem.
		% Source: https://en.wikipedia.org/wiki/Kirchhoff%27s_theorem
		% Test: stress-tests/graph/matrix-tree.cpp
		Create an $N\times N$ matrix \texttt{mat}, and for each edge $a \rightarrow b \in G$, do
		\texttt{mat[a][b]--, mat[b][b]++} (and \texttt{mat[b][a]--, mat[a][a]++} if $G$ is undirected).
		Remove the $i$th row and column and take the determinant; this yields the number of directed spanning trees rooted at $i$
		(if $G$ is undirected, remove any row/column).

	\subsection{Erdős–Gallai theorem}
		% Source: https://en.wikipedia.org/wiki/Erd%C5%91s%E2%80%93Gallai_theorem
		% Test: stress-tests/graph/erdos-gallai.cpp
		A simple graph with node degrees $d_1 \ge \dots \ge d_n$ exists iff $d_1 + \dots + d_n$ is even and for every $k = 1\dots n$,
		\[ \sum _{i=1}^{k}d_{i}\leq k(k-1)+\sum _{i=k+1}^{n}\min(d_{i},k). \]
